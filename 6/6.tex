%me=0 student solutions (ps file), me=1 - my solutions (sol file), me=2 - assignment (hw file)
\def\me{0}
\def\num{6}  %homework number
\def\due{Wednesday, December 14}  %due date
\def\course{CSCI-GA.1180-001 
} %course name, changed only once
\def\name{GOWTHAM GOLI (N17656180)}   %student changes (instructor keeps!)
%
\iffalse
INSTRUCTIONS: replace # by the homework number.
(if this is not ps#.tex, use the right file name)

  Clip out the ********* INSERT HERE ********* bits below and insert
appropriate TeX code.  Once you are done with your file, run

  ``latex ps#.tex''

from a UNIX prompt.  If your LaTeX code is clean, the latex will exit
back to a prompt.  To see intermediate results, type

  ``xdvi ps#.dvi'' (from UNIX prompt)
  ``yap ps#.dvi'' (if using MikTex in Windows)

after compilation. Once you are done, run

  ``dvips ps#.dvi''

which should print your file to the nearest printer.  There will be
residual files called ps#.log, ps#.aux, and ps#.dvi.  All these can be
deleted, but do not delete ps1.tex. To generate postscript file ps#.ps,
run

  ``dvips -o ps#.ps ps#.dvi''

I assume you know how to print .ps files (``lpr -Pprinter ps#.ps'')
\fi
%
\documentclass[11pt]{article}
\usepackage{amsfonts}
\usepackage{latexsym}
\usepackage[lined,boxed,linesnumbered]{algorithm2e}
\usepackage{amsmath}
\usepackage{amsthm}
\usepackage{array}
\usepackage{amssymb}
\usepackage{amsthm}
\usepackage{epsfig}
\usepackage{psfrag}
\usepackage{color}
\usepackage{tikz}
\usetikzlibrary{trees}
\usepackage{mathtools,xparse}
\usepackage{float}
\setlength{\oddsidemargin}{.0in}
\setlength{\evensidemargin}{.0in}
\setlength{\textwidth}{6.5in}
\setlength{\topmargin}{-0.4in}
\setlength{\textheight}{8.5in}
\usepackage{mathtools}
\DeclarePairedDelimiter\ceil{\lceil}{\rceil}
\DeclarePairedDelimiter\floor{\lfloor}{\rfloor}


\usepackage{listings}
\usepackage{color} %red, green, blue, yellow, cyan, magenta, black, white
\definecolor{mygreen}{RGB}{28,172,0} % color values Red, Green, Blue
\definecolor{mylilas}{RGB}{170,55,241}

\newtheorem{theorem}{Theorem}
\newcommand{\handout}[5]{
   \renewcommand{\thepage}{#1, Page \arabic{page}}
   \noindent
   \begin{center}
   \framebox{
      \vbox{
    \hbox to 5.78in { {\bf \course} \hfill #2 }
       \vspace{4mm}
       \hbox to 5.78in { {\Large \hfill #5  \hfill} }
       \vspace{2mm}
       \hbox to 5.78in { {\it #3 \hfill #4} }
      }
   }
   \end{center}
   \vspace*{4mm}
}

\newcounter{pppp}
\newcommand{\prob}{\arabic{pppp}}  %problem number
\newcommand{\increase}{\addtocounter{pppp}{1}}  %problem number

%first argument desription, second number of points
\newcommand{\newproblem}[2]{
\ifnum\me=0
\ifnum\prob>0 \newpage \fi
\increase
\setcounter{page}{1}
\handout{\name, Homework \num, Problem \arabic{pppp}}{\today}{Name: \name}{Due:
\due}{Solutions to Problem \prob\ of Homework \num\ }
\else
\increase
\section*{Problem \num-\prob~(#1) \hfill {#2}}
\fi
}

%\newcommand{\newproblem}[2]{\increase
%\section*{Problem \num-\prob~(#1) \hfill {#2}}
%}

\def\squarebox#1{\hbox to #1{\hfill\vbox to #1{\vfill}}}
\def\qed{\hspace*{\fill}
        \vbox{\hrule\hbox{\vrule\squarebox{.667em}\vrule}\hrule}}
\newenvironment{solution}{\begin{trivlist}\item[]{\bf Solution:}}
                      {\qed \end{trivlist}}
\newenvironment{solsketch}{\begin{trivlist}\item[]{\bf Solution Sketch:}}
                      {\qed \end{trivlist}}
\newenvironment{code}{\begin{tabbing}
12345\=12345\=12345\=12345\=12345\=12345\=12345\=12345\= \kill }
{\end{tabbing}}

%\newcommand{\eqref}[1]{Equation~(\ref{eq:#1})}

\newcommand{\hint}[1]{({\bf Hint}: {#1})}
%Put more macros here, as needed.
\newcommand{\room}{\medskip\ni}
\newcommand{\brak}[1]{\langle #1 \rangle}
\newcommand{\bit}[1]{\{0,1\}^{#1}}
\newcommand{\zo}{\{0,1\}}
\newcommand{\C}{{\cal C}}

\newcommand{\nin}{\not\in}
\newcommand{\set}[1]{\{#1\}}
\renewcommand{\ni}{\noindent}
\renewcommand{\gets}{\leftarrow}
\renewcommand{\to}{\rightarrow}
\newcommand{\assign}{:=}

\newcommand{\AND}{\wedge}
\newcommand{\OR}{\vee}

\newcommand{\Forr}{\mbox{\bf For }}
\newcommand{\To}{\mbox{\bf to }}
\newcommand{\Do}{\mbox{\bf Do }}
\newcommand{\Ifi}{\mbox{\bf If }}
\newcommand{\Thenn}{\mbox{\bf Then }}
\newcommand{\Elsee}{\mbox{\bf Else }}
\newcommand{\Whilee}{\mbox{\bf While }}
\newcommand{\Repeatt}{\mbox{\bf Repeat }}
\newcommand{\Until}{\mbox{\bf Until }}
\newcommand{\Returnn}{\mbox{\bf Return }}
\newcommand{\Swap}{\mbox{\bf Swap }}

\DeclarePairedDelimiter{\abs}{\lvert}{\rvert}
\DeclarePairedDelimiter{\norm}{\lVert}{\rVert}
\NewDocumentCommand{\normL}{ s O{} m }{%
  \IfBooleanTF{#1}{\norm*{#3}}{\norm[#2]{#3}}_{L_2(\Omega)}%
}

\begin{document}

\lstset{language=Matlab,%
    %basicstyle=\color{red},
    breaklines=true,%
    morekeywords={matlab2tikz},
    keywordstyle=\color{blue},%
    morekeywords=[2]{1}, keywordstyle=[2]{\color{black}},
    identifierstyle=\color{black},%
    stringstyle=\color{mylilas},
    commentstyle=\color{mygreen},%
    showstringspaces=false,%without this there will be a symbol in the places where there is a space
    numbers=left,%
    numberstyle={\tiny \color{black}},% size of the numbers
    numbersep=9pt, % this defines how far the numbers are from the text
    emph=[1]{for,end,break},emphstyle=[1]\color{red}, %some words to emphasise
    %emph=[2]{word1,word2}, emphstyle=[2]{style},    
}

\ifnum\me=0
%\handout{PS\num}{\today}{Name: **** INSERT YOU NAME HERE ****}{Due:
%\due}{Solutions to Problem Set \num}
%
%I collaborated with *********** INSERT COLLABORATORS HERE (INDICATING
%SPECIFIC PROBLEMS) *************.
\fi
\ifnum\me=1
\handout{PS\num}{\today}{Name: Yevgeniy Dodis}{Due: \due}{Solution
{\em Sketches} to Problem Set \num}
\fi
\ifnum\me=2
\handout{PS\num}{\today}{Lecturer: Yevgeniy Dodis}{Due: \due}{Problem
Set \num}
\fi

\newproblem{1}

\begin{itemize}
\item[(a)]
\ifnum\me<2
\begin{solution}

We can see this as $n = 5$ Bernoulli trials in which a success is when the toss produces a head. Since the coin is fair, the probability of success, $p = 1/2$. Therefore, the probability of getting $k = 3$ successes (producing a head) in $n = 5$ trails with success probability $ p = 1/2$ and failure probability $ q = 1-p = 1/2$ is 
\begin{align*}
\binom n k p^k q^{n-k} &= \binom 5 3 (\frac{1}{2})^3 (\frac{1}{2})^{5-3}\\
&= 0.3125
\end{align*}
\begin{lstlisting}
n = 5; k = 3; p = 0.5 ;
nchoosek(n,k)*p^k*(1-p)^(n-k)
\end{lstlisting}
\end{solution}
\fi

\item[(b)]
\ifnum\me<2
\begin{solution}

We can see this as $n = 40$ Bernoulli trials in which a success is when the toss produces a head. Since the coin is fair, the probability of success, $p = 1/2$. Therefore, the probability of getting $k = 20$ successes (producing a head) in $n = 40$ trails with success probability $ p = 1/2$ and failure probability $ q = 1-p = 1/2$ is 
\begin{align*}
\binom n k p^k q^{n-k} &= \binom {40} {20} (\frac{1}{2})^{20} (\frac{1}{2})^{40-20}\\
&= 0.1254
\end{align*}
\begin{lstlisting}
n = 40; k = 20; p = 0.5 ;
nchoosek(n,k)*p^k*(1-p)^(n-k)
\end{lstlisting}

\end{solution}
\fi

\item[(c)]
\ifnum\me<2
\begin{solution}
$\binom n k$ will be maximum for $k = \ceil{n/2}, \floor{n/2}$

We know that probability of $k$ heads out of $40$ tosses is $ p = \binom {40} {k} (\frac{1}{2})^{40} \implies p$ will be maximum when $\binom {40} k$ will be maximum $\implies \binom {40} k$ will be maximum for $k = 40/2 = 20$  
\end{solution}
\fi
\end{itemize}
\newproblem{2}

\begin{itemize}
\item[(1)] 

\ifnum\me<2
\begin{solution}\\

Let $h_1, h_2, h_3, h_4$ are the hats. Now, we have $4! = 24$ arrangements of the hats.  In one of the 24 arrangements, each person gets his/her own hat correctly. Let $(h_1, h_2, h_3, h_4)$ be that arrangement i.e., $h_i$ goes to person $i$.

$A_1$ is the event that person $1$ correctly gets $h_1$. This means this arrangement will have the form $(h_1, -, -, -)$. Now, we can arrange the remaining three hats in the three blanks in $3!$ ways. Therefore, the probability of $A_1$ is $\frac{3!}{4!} = 1/4$

To calculate $E[X_1]$, $X_1 = 1$ when $A_1$ occurs and $X_1 = 0$ otherwise.

$\therefore E[X_1] = 1 \times P[A_1] = 1/4$
\end{solution}
\fi

\item[(2)] 
\ifnum\me<2
\begin{solution}

$A_1 \cap A_2$ is the event that person $1$ correctly gets $h_1$ and person $2$ correctly gets $h_2$. This means this arrangement will have the form $(h_1, h_2, -, -)$. Now, we can arrange the remaining two hats in the two blanks in $2!$ ways. Therefore, the probability of $A_1 \cap A_2$ is $\frac{2!}{4!} = 1/12$

To calculate $E[X_1 \times X_2]$, $X_1 \times X_2 = 1$ only when $X_1 = 1, X_2 = 1$ i.e. when $A_1, A_2$ both happen and  $X_1 \times X_2 = 0$ otherwise.

$\therefore E[X_1 \times X_2] = 1 \times P[A_1 \cap A_2] = 1/12$
\end{solution}
\fi

\item[(3)] 
\ifnum\me<2
\begin{solution}
Consider the events $A_1, A_2$. It is easy to see that $P[A_2] = P[A_1] = 1/4 \implies P[A_1] \times P[A_2] = 1/4 \times 1/4 = 1/16$.

$P[A_1 \cap A_2] = 1/12 \neq P[A_1] \times P[A_2] \implies A_1, A_2$ are not independent events. 
\end{solution}
\fi
\end{itemize}


\newproblem{3}

\begin{itemize}
\item[(1)] 

\ifnum\me<2
\begin{solution}

Let 
\begin{itemize}
\item $R$ be the event that it rains.
\item $F$ be the event that there is a forecast of rain.
\item $U$ be the event that the professor brings an umbrella.
\end{itemize}
We need to find $P[U|R^c]$ and we have ,
\begin{align*}
P[R] &= 1/2\\
P[R|F] &= 2/3 \implies P[R^c|F] = 1/3\\
P[R^c|F^c] &= 2/3 \implies P[R|F^c] = 1/3\\
P[U|F] &= 1 \implies P[U^c|F] = 0\\
P[U|F^c] &= 1/3 \implies P[U^c|F^c] = 2/3\\
\therefore P[R] &= P[R|F] \cdot P[F] + P[R|F^c] \cdot P[F^c] \tag{1}
\end{align*}
Solve for $P[F]$ using $P[F^c] = 1-P[F]$ and substituting the above values in (1), we get, $P[F] = 1/2$.

The professor does not bring an umbrella to the office, given that
it rains that day is equivalent to 
\begin{itemize}
\item Given that it rains that day, the forecast is that there will be no rain and given that forecast is that there will be no rain, the professor does not bring an umbrella (or)
\item Given that it rains that day, the forecast is that there will be rain and given that forecast is that there will be rain, the professor does not bring an umbrella.
\end{itemize}
Mathematically we can express the same as follows
\begin{align*}
P[U^c|R] &= P[U^c|F^c] \cdot P[F^c|R] + P[U^c|F] \cdot P[F|R]\\
\implies P[U^c|R] &= 2/3 \times P[F^c|R] + 0 \times P[F|R] \tag{2}
\end{align*}
Now we need to calculate $P[F^c|R]$
\begin{align*}
P[F^c|R] &= \frac{P[R|F^c] \cdot P[F^c]}{P[R]} = \frac{\frac{1}{3} \times \frac{1}{2}}{\frac{1}{2}} = \frac{1}{3}
\end{align*}
Substituting $P[F^c|R]$ in (2) we get, $P[U^c|R] = 2/3 \times 1/3 = 2/9$
\end{solution}
\fi
\item[(2)] 
\ifnum\me<2
\begin{solution}

The professor brings an umbrella to the office, given that it does
not rain that day is equivalent to
\begin{itemize}
\item Given that it does not rain that day, the forecast is that there will be no rain and given that forecast is that there will be no rain, the professor brings an umbrella (or)
\item Given that it does not rain that day, the forecast is that there will be rain and given that forecast is that there will be rain, the professor  brings an umbrella.
\end{itemize}
Mathematically we can express the same as follows
\begin{align*}
P[U|R^c] &= P[U|F^c] \cdot P[F^c|R^c] + P[U|F] \cdot P[F|R^c] \\
\implies P[U|R^c] &= 1/3 \times P[F^c|R^c] + 1 \cdot P[F|R^c] \tag{3}
\end{align*}
Now we need to calculate $P[F|R^c]$ 
\begin{align*}
P[F|R^c] &= \frac{P[R^c|F] \cdot P[F]}{P[R^c]} = \frac{\frac{1}{3} \times \frac{1}{2}}{\frac{1}{2}} = \frac{1}{3} \implies P[F^c|R^c] = \frac{2}{3}
\end{align*}
Substituting $P[F|R^c], P[F^c|R^c$ in (2) we get, $P[U|R^c] = 1/3 \times 2/3 + 1 \times 1/3 = 5/9$

\end{solution}
\fi

\end{itemize}

\newproblem{4}

\begin{itemize}
\item[(a)] 

\ifnum\me<2
\begin{solution}

Since there are stocks exactly 5 strawberry pies and 5 cherry pies, every customer will receive his/her requested kind of pie only if 5 people want strawberry pie and the rest of the 5 people want cherry pies. This can be seen as $n = 10$ Bernoulli trials in which a success is when the 5 persons that requests the strawberry pie gets strawberry pie (which automatically means the rest of the 5 persons get cherry pie as requested). Thus, the probability of success, $p = 1/2$. Therefore, the probability of getting $k = 5$ successes in $n = 10$ trails with success probability $ p = 1/2$ and failure probability $ q = 1-p = 1/2$ is
\begin{align*}
\binom n k p^k q^{n-k} &= \binom {10} {5} (\frac{1}{2})^{5} (\frac{1}{2})^{10-5}\\
&= 0.2461
\end{align*}
\end{solution}
\fi
\item[(b)] 
\ifnum\me<2
\begin{solution}

Since there are stocks 8 strawberry pies and 8 cherry pies, every customer will receive his/her requested kind of pie only if 
\begin{itemize}
\item 2 people want strawberry pie and the rest of the 8 people want cherry pies $\implies k =2$.
\item 3 people want strawberry pie and the rest of the 7 people want cherry pies $\implies k =3$.
\item $\vdots$
\item 8 people want strawberry pie and the rest of the 2 people want cherry pies $\implies k =8$.
\end{itemize}
Summing up we get, probability that every customer will receive his/her requested kind of pie is 
\begin{align*}
& \binom n 2 p^2 q^{n-2} + \binom n 3 p^3 q^{n-3} + \dots + \binom n 8 p^8 q^{n-8}\\
&= \binom {10} 2 (1/2)^{10} + \binom {10} 3 (1/2)^{10} + \dots + \binom {10} 8 (1/2)^{10} \\
&= (1/2)^{10} (2 \binom {10} 2  + 2 \binom {10} 3 + 2 \binom {10} 4 + \binom {10} 5)\\
&= 0.9785
\end{align*}
\begin{lstlisting}
n = 10; p = 0.5;
(p^10)*(2*nchoosek(n,2)+2*nchoosek(n,3)
+2*nchoosek(n,4)+nchoosek(n,5))
\end{lstlisting}
\end{solution}
\fi

\end{itemize}

\newproblem{5}

\begin{itemize}
\item[(1)] 
\ifnum\me<2
\begin{solution}
Let $\{X=k\}$ denote that heads has been tossed $k$ times in 3 chances. The gambler
will win when $\{X=1\}$ i.e., if he tosses either $H/TH/TTH$ and the gambler will lose when $\{X=0\}$ i.e., if he tosses $TTT$. 
\begin{align*}
P(H) &= 1/2, P(TH) = 1/4, P(TTH) = 1/8\\
\implies P_X(1) &= 1/2 + 1/4 + 1/8 = 7/8\\
P(TTT) &= 1/8\\
\implies P_X(0) &= 1/8\\
\therefore E[X] &= (0 \times P_X(0)) + (1 \times P_X(1))  = 7/8\\
\end{align*}
\begin{align*}
E[X^2] &= (0^2 \times P_X(0)) + (1^2 \times P_X(1))  = 7/8\\
\therefore var[X] &= E[X^2] - (E[X])^2 = 7/8 - (7/8)^2 = 7/64\\
\end{align*}

\end{solution}
\fi

\item[(2)]
\ifnum\me<2
\begin{solution} 
Let $\{Y=k\}$ denote that tails has been tossed $k$ times in 3 chances.
\begin{align*}
P_Y(1) &= P(TH) = 1/4\\
P_Y(2) &= P(TTH) = 1/8 \\
P_Y(3) &= P(TTT) = 1/8\\
\therefore E[Y] &= (0 \times P_Y(0)) + (1 \times P_Y(1)) + (2 \times P_Y(2)) + (3 \times P_Y(3))\\
&= (1 \times 1/4) + (2 \times 1/8) + (3 \times 1/8) = 7/8
\end{align*}

\begin{align*}
E[Y^2] &= (0^2 \times P_Y(0)) + (1^2 \times P_Y(1)) + (2^2 \times P_Y(2)) + (3^2 \times P_Y(3))   = 15/8\\
\therefore var[X] &= E[X^2] - (E[X])^2 = 15/8 - (7/8)^2 = 71/64\\
\end{align*}

\end{solution}
\fi

\end{itemize}
\end{document}


