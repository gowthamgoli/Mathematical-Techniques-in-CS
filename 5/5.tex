%me=0 student solutions (ps file), me=1 - my solutions (sol file), me=2 - assignment (hw file)
\def\me{0}
\def\num{5}  %homework number
\def\due{Tuesday, November 29}  %due date
\def\course{CSCI-GA.1180-001 
} %course name, changed only once
\def\name{GOWTHAM GOLI (N17656180)}   %student changes (instructor keeps!)
%
\iffalse
INSTRUCTIONS: replace # by the homework number.
(if this is not ps#.tex, use the right file name)

  Clip out the ********* INSERT HERE ********* bits below and insert
appropriate TeX code.  Once you are done with your file, run

  ``latex ps#.tex''

from a UNIX prompt.  If your LaTeX code is clean, the latex will exit
back to a prompt.  To see intermediate results, type

  ``xdvi ps#.dvi'' (from UNIX prompt)
  ``yap ps#.dvi'' (if using MikTex in Windows)

after compilation. Once you are done, run

  ``dvips ps#.dvi''

which should print your file to the nearest printer.  There will be
residual files called ps#.log, ps#.aux, and ps#.dvi.  All these can be
deleted, but do not delete ps1.tex. To generate postscript file ps#.ps,
run

  ``dvips -o ps#.ps ps#.dvi''

I assume you know how to print .ps files (``lpr -Pprinter ps#.ps'')
\fi
%
\documentclass[11pt]{article}
\usepackage{amsfonts}
\usepackage{latexsym}
\usepackage[lined,boxed,linesnumbered]{algorithm2e}
\usepackage{amsmath}
\usepackage{amsthm}
\usepackage{array}
\usepackage{amssymb}
\usepackage{amsthm}
\usepackage{epsfig}
\usepackage{psfrag}
\usepackage{color}
\usepackage{tikz}
\usetikzlibrary{trees}
\usepackage{mathtools,xparse}
\usepackage{float}
\setlength{\oddsidemargin}{.0in}
\setlength{\evensidemargin}{.0in}
\setlength{\textwidth}{6.5in}
\setlength{\topmargin}{-0.4in}
\setlength{\textheight}{8.5in}


\usepackage{listings}
\usepackage{color} %red, green, blue, yellow, cyan, magenta, black, white
\definecolor{mygreen}{RGB}{28,172,0} % color values Red, Green, Blue
\definecolor{mylilas}{RGB}{170,55,241}

\newtheorem{theorem}{Theorem}
\newcommand{\handout}[5]{
   \renewcommand{\thepage}{#1, Page \arabic{page}}
   \noindent
   \begin{center}
   \framebox{
      \vbox{
    \hbox to 5.78in { {\bf \course} \hfill #2 }
       \vspace{4mm}
       \hbox to 5.78in { {\Large \hfill #5  \hfill} }
       \vspace{2mm}
       \hbox to 5.78in { {\it #3 \hfill #4} }
      }
   }
   \end{center}
   \vspace*{4mm}
}

\newcounter{pppp}
\newcommand{\prob}{\arabic{pppp}}  %problem number
\newcommand{\increase}{\addtocounter{pppp}{1}}  %problem number

%first argument desription, second number of points
\newcommand{\newproblem}[2]{
\ifnum\me=0
\ifnum\prob>0 \newpage \fi
\increase
\setcounter{page}{1}
\handout{\name, Homework \num, Problem \arabic{pppp}}{\today}{Name: \name}{Due:
\due}{Solutions to Problem \prob\ of Homework \num\ }
\else
\increase
\section*{Problem \num-\prob~(#1) \hfill {#2}}
\fi
}

%\newcommand{\newproblem}[2]{\increase
%\section*{Problem \num-\prob~(#1) \hfill {#2}}
%}

\def\squarebox#1{\hbox to #1{\hfill\vbox to #1{\vfill}}}
\def\qed{\hspace*{\fill}
        \vbox{\hrule\hbox{\vrule\squarebox{.667em}\vrule}\hrule}}
\newenvironment{solution}{\begin{trivlist}\item[]{\bf Solution:}}
                      {\qed \end{trivlist}}
\newenvironment{solsketch}{\begin{trivlist}\item[]{\bf Solution Sketch:}}
                      {\qed \end{trivlist}}
\newenvironment{code}{\begin{tabbing}
12345\=12345\=12345\=12345\=12345\=12345\=12345\=12345\= \kill }
{\end{tabbing}}

%\newcommand{\eqref}[1]{Equation~(\ref{eq:#1})}

\newcommand{\hint}[1]{({\bf Hint}: {#1})}
%Put more macros here, as needed.
\newcommand{\room}{\medskip\ni}
\newcommand{\brak}[1]{\langle #1 \rangle}
\newcommand{\bit}[1]{\{0,1\}^{#1}}
\newcommand{\zo}{\{0,1\}}
\newcommand{\C}{{\cal C}}

\newcommand{\nin}{\not\in}
\newcommand{\set}[1]{\{#1\}}
\renewcommand{\ni}{\noindent}
\renewcommand{\gets}{\leftarrow}
\renewcommand{\to}{\rightarrow}
\newcommand{\assign}{:=}

\newcommand{\AND}{\wedge}
\newcommand{\OR}{\vee}

\newcommand{\Forr}{\mbox{\bf For }}
\newcommand{\To}{\mbox{\bf to }}
\newcommand{\Do}{\mbox{\bf Do }}
\newcommand{\Ifi}{\mbox{\bf If }}
\newcommand{\Thenn}{\mbox{\bf Then }}
\newcommand{\Elsee}{\mbox{\bf Else }}
\newcommand{\Whilee}{\mbox{\bf While }}
\newcommand{\Repeatt}{\mbox{\bf Repeat }}
\newcommand{\Until}{\mbox{\bf Until }}
\newcommand{\Returnn}{\mbox{\bf Return }}
\newcommand{\Swap}{\mbox{\bf Swap }}

\DeclarePairedDelimiter{\abs}{\lvert}{\rvert}
\DeclarePairedDelimiter{\norm}{\lVert}{\rVert}
\NewDocumentCommand{\normL}{ s O{} m }{%
  \IfBooleanTF{#1}{\norm*{#3}}{\norm[#2]{#3}}_{L_2(\Omega)}%
}

\begin{document}

\lstset{language=Matlab,%
    %basicstyle=\color{red},
    breaklines=true,%
    morekeywords={matlab2tikz},
    keywordstyle=\color{blue},%
    morekeywords=[2]{1}, keywordstyle=[2]{\color{black}},
    identifierstyle=\color{black},%
    stringstyle=\color{mylilas},
    commentstyle=\color{mygreen},%
    showstringspaces=false,%without this there will be a symbol in the places where there is a space
    numbers=left,%
    numberstyle={\tiny \color{black}},% size of the numbers
    numbersep=9pt, % this defines how far the numbers are from the text
    emph=[1]{for,end,break},emphstyle=[1]\color{red}, %some words to emphasise
    %emph=[2]{word1,word2}, emphstyle=[2]{style},    
}

\ifnum\me=0
%\handout{PS\num}{\today}{Name: **** INSERT YOU NAME HERE ****}{Due:
%\due}{Solutions to Problem Set \num}
%
%I collaborated with *********** INSERT COLLABORATORS HERE (INDICATING
%SPECIFIC PROBLEMS) *************.
\fi
\ifnum\me=1
\handout{PS\num}{\today}{Name: Yevgeniy Dodis}{Due: \due}{Solution
{\em Sketches} to Problem Set \num}
\fi
\ifnum\me=2
\handout{PS\num}{\today}{Lecturer: Yevgeniy Dodis}{Due: \due}{Problem
Set \num}
\fi

\newproblem{1}

\begin{itemize}
\item[(1)]
\ifnum\me<2
\begin{solution}
\begin{equation*}
20 = 5 \times 4
\end{equation*}
This is possible under the circumstances that
\begin{itemize}
\item the order of paths matter
\item Any of the 5 paths can be chosen on the trip from bottom to top.
\item On the trip from top to bottom, we can't choose the same path as we chose on the trip to top, so we have only 4 choices.
\item Therefore, the total number of choices = 5 $\times$ 4
\end{itemize}
Acceptable round trips are $\{(1,2),(1,3),(1,4),(1,5),(2,1),(2,3),(2,4),(2,5),(3,1),(3,2),(3,4),\\(3,5),(4,1),(4,2),(4,3),(4,5),(5,1),(5,2),(5,3),(5,4)\}$
\end{solution}
\fi

\item[(2)]
\ifnum\me<2
\begin{solution}
\begin{equation*}
25 = 5 \times 5
\end{equation*}
This is possible under the circumstances that
\begin{itemize}
\item the order of paths matter
\item Any of the 5 paths can be chosen on the trip from bottom to top.
\item Any of the 5 paths can be chosen on the trip from top to bottom.
\item Therefore, the total number of choices = 5 $\times$ 5
\end{itemize}
Acceptable round trips are $\{(1,1),(1,2),(1,3),(1,4),(1,5),(2,1),(2,2),(2,3),(2,4),(2,5),(3,1),\\(3,2),(3,3),(3,4),(3,5),(4,1),(4,2),(4,3),(4,4),(4,5),(5,1),(5,2),(5,3),(5,4),(5,5)\}$
\end{solution}
\fi

\item[(3)]
\ifnum\me<2
\begin{solution}
\begin{equation*}
\binom 5 2 = 10
\end{equation*}
It is clear that, we are any 2 paths from the given 5 paths. Therefore, this is possible under the circumstances that
\begin{itemize}
\item the order of the paths don't matter.
\item the path from bottom to top is different from the path from bottom to top i.e., there are no repetitions.
\end{itemize}
Acceptable round trips are $\{(1,2),(1,3),(1,4),(1,5),(2,3),(2,4),(2,5),(3,4),(3,5),(4,5)\}$
\end{solution}
\fi

\item[(4)]
\ifnum\me<2
\begin{solution}
\begin{equation*}
\binom 6 2 = 15
\end{equation*}
This is in the form of $\binom {n+k-1} {k}$ where $n=5, k=2$. Therefore, this is possible under the circumstances that
\begin{itemize}
\item the order of the paths don't matter.
\item the path from bottom to top and top to bottom could be equal i.e., repetitions are allowed
\end{itemize}
Acceptable round trips are $\{(1,2),(1,2),(1,3),(1,4),(1,5),(2,2),(2,3),(2,4),(2,5),(3,3),(3,4),\\(3,5),(4,4),(4,5),(5,5)\}$
\end{solution}
\fi
\end{itemize}
\newproblem{2}

\begin{itemize}
\item[(1)] 

\ifnum\me<2
\begin{solution}

Let $b$, $g$ represent boy and girl, then we have 4 possibilities as follows: $\{(b,b),(b,g),(g,g),(g,b)\}$ where the first element in the tuple is the younger child and the second element in the older child. 

Let $C$ be the event that the both children are girls. Therefore, we need to compute $P[C | A]$.
\begin{equation*}
P[C | A] = \frac{P[C \cap A]}{P[A]}
\end{equation*}
\begin{itemize}
\item $P[A]$ = Probability the older child is a girl. In the above set we can see that $g$ occurs at the second position in the tuple twice $\implies P[A] = 2/4 = 1/2$
\item $C \cap A$ = Both the children are girls and the older child is a girl. If $C$ happens then $A$ happens automatically $\implies C \cap A = C \therefore$ From the above set we can see that, $P[C] = 1/4$ 
\end{itemize}
\begin{equation*}
P[C | A] = \frac{1/4}{1/2} = 1/2
\end{equation*}
\end{solution}
\fi

\item[(2)] 
\ifnum\me<2
\begin{solution}
Let $C$ be the event that both the children are girls and $D$ be the event that atleast one of the children is a girl. Therefore, we need to compute $P[C|D]$.
\begin{equation*}
P[C | D] = \frac{P[C \cap D]}{P[D]}
\end{equation*}
\begin{itemize}
\item $P[D]$ = Probability the atleast one of the children is a girl. In the above set we can see that $g$ belongs to a tuple 3 times. $\implies P[D] = 3/4$
\item $C \cap D$ = Both the children are girls and atleast one of the children is a girl. If $C$ happens then $D$ happens automatically $\implies C \cap D = C \therefore$ From the above set we can see that, $P[C] = 1/4$ 
\end{itemize}
\begin{equation*}
P[C | D] = \frac{1/4}{3/4} = 1/3
\end{equation*}
\end{solution}

\fi
\end{itemize}


\newproblem{3}

\begin{itemize}
\item[(a)] 

\ifnum\me<2
\begin{solution}
Using the formula in the lecture notes, we know that, the probability of $k$ successes in $n$ trials where the success probability is $p$, failure probability is $q = 1-p$ is
\begin{equation*}
\binom n k p^k q^{n-k}
\end{equation*}
In this case, we have $n = 11$ (total number of carrots), $k = 11$ (number of acceptable carrots), $q = 0.14$ (unacceptable carrot probability), $p = 1-0.14 = 0.86$. Let $A$ be the event that all the carrots in the sample will be sweet (or no carrot in the sample is sour), then
\begin{align*}
P[A] = \binom {11} {11} \times 0.86^{11} \times 0.14^{(11-11)} = 0.1903
\end{align*}
\end{solution}
\fi
\item[(b)] 
\ifnum\me<2
\begin{solution}
Let $B$ be the event that 1 or more carrots in the sample is sour, then $B = A^c$
\begin{equation*}
P[B] = P[A^c] = 1-P[A] = 1-0.1903 = 0.8097
\end{equation*}
\end{solution}
\fi

\item[(c)] 
\ifnum\me<2
\begin{solution}
Let $C$ be the event that at most 1 carrot in the sample is sour $\implies$ 0 carrots in the sample is sour or exactly 1 carrot in the sample is sour. We know from part(a), $P[A] = $ the probability of 0 carrots in the sample is sour is 0.1905. Let $D$ be the event that  exactly 1 carrot in the sample is sour (or 10 carrots are sweet), then similar to part (a), we get
\begin{align*}
P[D] &= \binom {11} {1} \times 0.86^{10} \times 0.14^{(11-10)} = 0.3408\\
\therefore P[C] &= P[A] + P[D] = 0.1905 + 0.3408 = 0.5313
\end{align*}
\end{solution}
\fi

\end{itemize}


\newproblem{4}

\ifnum\me<2
\begin{solution}
\begin{lstlisting}[frame=single]
format short e;
p = 0.75;
n = 12;
z = zeros(1,n);
x = zeros(1,n);
for i=1:n
    z(i) = rand
    if (z(i) <= p)
        x(i) = 1;
    else
        x(i) = 0;
    end
end
\end{lstlisting}

\begin{quote}
\textbf{Trial 1:}

\texttt{
z =\\
  Columns 1 through 8\\
   1.1921e-01   9.3983e-01   6.4555e-01   4.7946e-01   6.3932e-01   5.4472e-01   6.4731e-01   5.4389e-01\\ 
  Columns 9 through 12\\
   7.2105e-01   5.2250e-01   9.9370e-01   2.1868e-01\\
   x =\\
     1     0     1     1     1     1     1     1     1     1     0     1 = 10 heads = 0.83 
}

\textbf{Trial 2:}

\texttt{
z =\\
  Columns 1 through 7\\
   1.0580e-01   1.0970e-01   6.3591e-02   4.0458e-01   4.4837e-01   3.6582e-01   7.6350e-01\\
  Columns 8 through 12\\
   6.2790e-01   7.7198e-01   9.3285e-01   9.7274e-01   1.9203e-01\\
x =\\
     1     1     1     1     1     1     0     1     0     0     0     1 = 8 heads = 0.66}
\pagebreak

\textbf{Trial 3:}

\texttt{
z =\\
  Columns 1 through 7\\
   7.0405e-01   4.4231e-01   1.9578e-02   3.3086e-01   4.2431e-01   2.7027e-01   1.9705e-01\\
  Columns 8 through 12\\
   8.2172e-01   4.2992e-01   8.8777e-01   3.9118e-01   7.6911e-01\\
x =\\
     1     1     1     1     1     1     1     0     1     0     1     0 = 9 heads = 0.75
}

\textbf{Trial 4:}

\texttt{
z =\\  Columns 1 through 7\\
   4.7952e-01   8.0135e-01   2.2784e-01   4.9809e-01   9.0085e-01   5.7466e-01   8.4518e-01\\
  Columns 8 through 12\\
   7.3864e-01   5.8599e-01   2.4673e-01   6.6642e-01   8.3483e-02\\
x =\\
     1     0     1     1     0     1     0     1     1     1     1     1 = 9 heads = 0.75
}

\textbf{Trial 5:}

\texttt{
z =\\
  Columns 1 through 7\\
   6.2596e-01   6.6094e-01   7.2975e-01   8.9075e-01   9.8230e-01   7.6903e-01   5.8145e-01\\
  Columns 8 through 12\\
   9.2831e-01   5.8009e-01   1.6983e-02   1.2086e-01   8.6271e-01\\
x =\\
     1     1     1     0     0     0     1     0     1     1     1     0 = 7 heads = 0.58 
}

\textbf{Trial 6:}

\texttt{
z =\\
  Columns 1 through 8\\
   4.8430e-01   8.4486e-01   2.0941e-01   5.5229e-01   6.2988e-01   3.1991e-02   6.1471e-01   3.6241e-01\\
  Columns 9 through 12\\
   4.9533e-02   4.8957e-01   1.9251e-01   1.2308e-01\\
x =\\
     1     0     1     1     1     1     1     1     1     1     1     1 = 11 heads = 0.91
}

\textbf{Trial 7:}

\texttt{
z =\\ Columns 1 through 8\\
   4.9036e-01   8.5300e-01   8.7393e-01   2.7029e-01   2.0846e-01   5.6498e-01   6.4031e-01   4.1703e-01\\
  Columns 9 through 12\\
   2.0598e-01   9.4793e-01   8.2071e-02   1.0571e-01\\
x =\\
     1     0     0     1     1     1     1     1     1     0     1     1 = 9 heads = 0.75
}

\textbf{Trial 8:}

\texttt{
z =\\
  Columns 1 through 8\\
   1.4204e-01   1.6646e-01   6.2096e-01   5.7371e-01   5.2078e-02   9.3120e-01   7.2866e-01   7.3784e-01\\
  Columns 9 through 12\\
   6.3405e-02   8.6044e-01   9.3441e-01   9.8440e-01\\
x =\\
     1     1     1     1     1     0     1     1     1     0     0     0
 = 9 heads = 0.75
}
\end{quote}
We can see that in trials 3,4,7,8 the ratio of success is exactly equal to 0.75 and in the rest of the trails, it is very close to 0.75
\end{solution}
\fi

\newproblem{5}

\begin{itemize}
\item[(a)] 
\ifnum\me<2
\begin{solution}
This is similar to linear least squares problem of minimizing $\norm{Ax-b}^2_2$ where the $m$ vector $e$ corresponds to $m \times n$ matrix $A$ and the scalar $\beta$ corresponds to  the $n$ vector $x$.  $ \therefore e^Te\beta = e^Tb \implies \beta = \frac{e^Tb}{e^Te}$. $e^Te = m, e^Tb = b_1+\dots+b_m \implies \beta = \frac{b_1+\dots+b_m}{m}$
\end{solution}
\fi

\item[(b)]
\ifnum\me<2
\begin{solution} The maximum absolute value of an element is the infinity norm of a vector. Therefore, we need to choose $\beta$ such that $|b_i - \beta|$ is overall minimized. Intuitively, this will happen when $\beta$ is mid-way between the maximum ($b_{max}$) and minimum ($b_{min}$) elements of $b$. This can be heuristically justified as follows
\begin{itemize}
\item If we choose $\beta$ closer to $b_{min}$ then $|b_i - \beta|$ will become too big  $\forall b_i$ that are closer to $b_{max}$.
\item If we choose $\beta$ closer to $b_{max}$ then $|b_i - \beta|$ will become too big  $\forall b_i$ that are closer to $b_{min}$.
\end{itemize}
Hence $\beta$ must lie in the mid way between $b_{min}$ and $b_{max}$ i.e., $\beta = \frac{b_{min}+b_{max}}{2}$
\end{solution}
\fi

%\item[(c)]
%\ifnum\me<2
%\begin{solution}
%\end{solution}
%\fi

\item[(d)]
Using the code below for each part in (i), (ii), (iii) we get, 
\begin{lstlisting}[frame=single]
b = %Insert value of b here for each part
e = [1 1 1 1 1 1 1];
beta_one_norm = median(b)
beta_two_norm = sum(b)/7
beta_inf_norm = (min(b) + max(b))/2

r1 = b-beta_one_norm*e
norm_r1 = norm(r1,1)

r2 = b-beta_two_norm*e
norm_r2 = norm(r2,2)

rinf = b-beta_inf_norm*e
norm_rinf = norm(rinf,inf)
\end{lstlisting}
\begin{itemize}
\item[(i)]
\ifnum\me<2
\begin{solution}
\begin{align*}
\beta_1 &= \beta_2 = \beta_{\infty} = 0\\
r_1 &= \begin{pmatrix}
-150  &  25   &  0  & -70   & 70  & 150 &  -25
\end{pmatrix}^T\\
\norm{r_1}_1 &= 490\\
r_2 &=  \begin{pmatrix} -150  &  25   &  0 &  -70   & 70  & 150  & -25 \end{pmatrix}^T\\
\norm{r_2}_2 &= 236.7488\\
r_{\infty} &= \begin{pmatrix}  -150  &   25   &   0  &  -70 &    70  &  150  &  -25 \end{pmatrix}^T\\
\norm{r_{\infty}} &= 150
\end{align*}
$\beta_1 = 0 \implies$ The median of $b$ is 0. Therefore the minimum value of $b$ must be less than or equal to 0 and $\beta_{\infty} = 0 \implies$ that $b_{\max} = -b_{\min}$ and $\beta_2 = 0 \implies$ the sum of all elements of $b$ is $0$. This could be possible when the elements of $b$ are symmetric around 0.
\end{solution}
\fi

\item[(ii)]
\ifnum\me<2
\begin{solution}
\begin{align*}
\beta_1 &= 1\\
\beta_2 &= 150.5714\\
\beta_{\infty} &= 250.5000\\
r_1 &= \begin{pmatrix}
0 &  499   &  0 &  249    & 0   &299 &    0
\end{pmatrix}^T\\
\norm{r_1} &= 1047\\
r_2 &=  \begin{pmatrix} -149.5714 & 349.4286 & -149.5714  & 99.4286 & -149.5714 & 149.4286 & -149.5714 \end{pmatrix}^T\\
\norm{r_2}_2 &= 493.7628\\
r_{\infty} &= \begin{pmatrix}  -249.5000 & 249.5000 & -249.5000 &  -0.5000 & -249.5000 &49.5000 & -249.5000 \end{pmatrix}^T\\
\norm{r_{\infty}} &= 249.5000
\end{align*}
$\beta_1 = 1 \implies$ The median of $b$ is 1. Therefore the minimum value of $b$ must be less than or equal to 1 and $\beta_{\infty} \approx 250 \implies$ the maximum value of $b$ must be greater than or equal to 499 and $\beta_2 \approx 150 \implies$ the sum of all elements of $b$ is $ 150 \times 7 = 750$ 
\end{solution}
\fi

\item[(iii)]
\ifnum\me<2
\begin{solution}
\begin{align*}
\beta_1 &= 1\\
\beta_2 &= 14285\\
\beta_{\infty} &= 49995\\
r_1 &= \begin{pmatrix}
 0     &     -3      &   -11       &    3   & 6     &     -6     &   99999
\end{pmatrix}^T\\
\norm{r_1} &= 100028\\
r_2 &= \begin{pmatrix}   -14284   &   -14287      -14295   &   -14281  &14278   &   -14290      & 85715 \end{pmatrix}^T\\
\norm{r_2}_2 &= 9.2583e+04\\
r_{\infty} &= \begin{pmatrix}   -49994      & -49997    &   -50005   &    -49991 &  -49988      & -50000    &    59005 \end{pmatrix}^T\\
\norm{r_{\infty}} &= 59005
\end{align*}
$\beta_1 = 1 \implies$ The median of $b$ is 1. Therefore the minimum value of $b$ must be less than or equal to 1 and $\beta_{\infty} = 49995 \implies$  $b_{max}$ must be large and $\beta_2 = 14285 \implies$ $b_{max}$ must have been contributed to the overall increase of the sum of the elements of $b$. 
\end{solution}
\fi
\end{itemize}


\item[(e)]
\ifnum\me<2
\begin{solution}
From part (iii) in the above question, we can see that $\beta_2$ and $\beta_{\infty}$ will be affected the most due to outliers and $\beta_1$ has no significant affect because
\begin{itemize}
\item $\beta_1$ depends only on the median.
\item $\beta_2$ depends on the average sum of the elements of $b$. So it can be significantly increased or decreased with large or small outliers.
\item $\beta_{\infty}$ depends on the maximum and minimum elements of $b$.

\end{itemize}
\end{solution}
\fi

\end{itemize}
\end{document}


