%me=0 student solutions (ps file), me=1 - my solutions (sol file), me=2 - assignment (hw file)
\def\me{0}
\def\num{1}  %homework number
\def\due{Tuesday, September 26}  %due date
\def\course{CSCI-GA.1180-001 
} %course name, changed only once
\def\name{GOWTHAM GOLI (N17656180)}   %student changes (instructor keeps!)
%
\iffalse
INSTRUCTIONS: replace # by the homework number.
(if this is not ps#.tex, use the right file name)

  Clip out the ********* INSERT HERE ********* bits below and insert
appropriate TeX code.  Once you are done with your file, run

  ``latex ps#.tex''

from a UNIX prompt.  If your LaTeX code is clean, the latex will exit
back to a prompt.  To see intermediate results, type

  ``xdvi ps#.dvi'' (from UNIX prompt)
  ``yap ps#.dvi'' (if using MikTex in Windows)

after compilation. Once you are done, run

  ``dvips ps#.dvi''

which should print your file to the nearest printer.  There will be
residual files called ps#.log, ps#.aux, and ps#.dvi.  All these can be
deleted, but do not delete ps1.tex. To generate postscript file ps#.ps,
run

  ``dvips -o ps#.ps ps#.dvi''

I assume you know how to print .ps files (``lpr -Pprinter ps#.ps'')
\fi
%
\documentclass[11pt]{article}
\usepackage{amsfonts}
\usepackage{latexsym}
\usepackage[lined,boxed,linesnumbered]{algorithm2e}
\usepackage{amsmath}
\usepackage{amsthm}
\usepackage{array}
\usepackage{amssymb}
\usepackage{amsthm}
\usepackage{epsfig}
\usepackage{psfrag}
\usepackage{color}
\usepackage{tikz}
\usetikzlibrary{trees}
\usepackage{mathtools}
\usepackage{float}
\setlength{\oddsidemargin}{.0in}
\setlength{\evensidemargin}{.0in}
\setlength{\textwidth}{6.5in}
\setlength{\topmargin}{-0.4in}
\setlength{\textheight}{8.5in}

\newtheorem{theorem}{Theorem}
\newcommand{\handout}[5]{
   \renewcommand{\thepage}{#1, Page \arabic{page}}
   \noindent
   \begin{center}
   \framebox{
      \vbox{
    \hbox to 5.78in { {\bf \course} \hfill #2 }
       \vspace{4mm}
       \hbox to 5.78in { {\Large \hfill #5  \hfill} }
       \vspace{2mm}
       \hbox to 5.78in { {\it #3 \hfill #4} }
      }
   }
   \end{center}
   \vspace*{4mm}
}

\newcounter{pppp}
\newcommand{\prob}{\arabic{pppp}}  %problem number
\newcommand{\increase}{\addtocounter{pppp}{1}}  %problem number

%first argument desription, second number of points
\newcommand{\newproblem}[2]{
\ifnum\me=0
\ifnum\prob>0 \newpage \fi
\increase
\setcounter{page}{1}
\handout{\name, Homework \num, Problem \arabic{pppp}}{\today}{Name: \name}{Due:
\due}{Solutions to Problem \prob\ of Homework \num\ }
\else
\increase
\section*{Problem \num-\prob~(#1) \hfill {#2}}
\fi
}

%\newcommand{\newproblem}[2]{\increase
%\section*{Problem \num-\prob~(#1) \hfill {#2}}
%}

\def\squarebox#1{\hbox to #1{\hfill\vbox to #1{\vfill}}}
\def\qed{\hspace*{\fill}
        \vbox{\hrule\hbox{\vrule\squarebox{.667em}\vrule}\hrule}}
\newenvironment{solution}{\begin{trivlist}\item[]{\bf Solution:}}
                      {\qed \end{trivlist}}
\newenvironment{solsketch}{\begin{trivlist}\item[]{\bf Solution Sketch:}}
                      {\qed \end{trivlist}}
\newenvironment{code}{\begin{tabbing}
12345\=12345\=12345\=12345\=12345\=12345\=12345\=12345\= \kill }
{\end{tabbing}}

%\newcommand{\eqref}[1]{Equation~(\ref{eq:#1})}

\newcommand{\hint}[1]{({\bf Hint}: {#1})}
%Put more macros here, as needed.
\newcommand{\room}{\medskip\ni}
\newcommand{\brak}[1]{\langle #1 \rangle}
\newcommand{\bit}[1]{\{0,1\}^{#1}}
\newcommand{\zo}{\{0,1\}}
\newcommand{\C}{{\cal C}}

\newcommand{\nin}{\not\in}
\newcommand{\set}[1]{\{#1\}}
\renewcommand{\ni}{\noindent}
\renewcommand{\gets}{\leftarrow}
\renewcommand{\to}{\rightarrow}
\newcommand{\assign}{:=}

\newcommand{\AND}{\wedge}
\newcommand{\OR}{\vee}

\newcommand{\Forr}{\mbox{\bf For }}
\newcommand{\To}{\mbox{\bf to }}
\newcommand{\Do}{\mbox{\bf Do }}
\newcommand{\Ifi}{\mbox{\bf If }}
\newcommand{\Thenn}{\mbox{\bf Then }}
\newcommand{\Elsee}{\mbox{\bf Else }}
\newcommand{\Whilee}{\mbox{\bf While }}
\newcommand{\Repeatt}{\mbox{\bf Repeat }}
\newcommand{\Until}{\mbox{\bf Until }}
\newcommand{\Returnn}{\mbox{\bf Return }}
\newcommand{\Swap}{\mbox{\bf Swap }}

\begin{document}

\ifnum\me=0
%\handout{PS\num}{\today}{Name: **** INSERT YOU NAME HERE ****}{Due:
%\due}{Solutions to Problem Set \num}
%
%I collaborated with *********** INSERT COLLABORATORS HERE (INDICATING
%SPECIFIC PROBLEMS) *************.
\fi
\ifnum\me=1
\handout{PS\num}{\today}{Name: Yevgeniy Dodis}{Due: \due}{Solution
{\em Sketches} to Problem Set \num}
\fi
\ifnum\me=2
\handout{PS\num}{\today}{Lecturer: Yevgeniy Dodis}{Due: \due}{Problem
Set \num}
\fi

\newproblem{1}

\begin{itemize}
 \item[(a)] 
 
\ifnum\me<2
\begin{solution}

Let, 
\(
A =
  \begin{pmatrix}
    a_{11} & a_{12}\\
    a_{21} & a_{22}
  \end{pmatrix} \implies
A^2 = \begin{pmatrix}
    a_{11}^2+a_{12}a_{21} & a_{11}a_{12}+a_{12}a_{22}\\
    a_{21}a_{11}+a_{22}a_{21} & a_{21}a_{12}+a_{22}^2
  \end{pmatrix} =
\begin{pmatrix}
    -1 & 0\\
    0 & -1
  \end{pmatrix}
\)

Therefore, 
\begin{align}
 &a_{11}^2+a_{12}a_{21} = -1 \\
 &a_{11}a_{12} = -a_{22}a_{12}\\
 &a_{11}a_{21} = -a_{22}a_{21}\\
 &a_{22}^2 + a_{21}a_{12} = -1
\end{align}
From equations (1) and (4), it is clear that $a_{12} \neq 0$ and $a_{21} \neq 0$. Therefore from equations (2) and (4), we can conclude that $a_{11} = -a_{22}$.
Let $a_{11} = 1, a_{12} = 1 \implies a_{22} = -1$ and $a_{21} = -2$\\ Therefore    \(
A =
  \begin{pmatrix}
    1 & 1\\
    -2 & -1
  \end{pmatrix}
\)
\end{solution}
\fi

\item[(b)] 
\ifnum\me<2
\begin{solution}
Let, 
\(
B =
  \begin{pmatrix}
    b_{11} & b_{12}\\
    b_{21} & b_{22}
  \end{pmatrix} \implies
B^2 = \begin{pmatrix}
    b_{11}^2+b_{12}b_{21} & b_{11}b_{12}+b_{12}b_{22}\\
    b_{21}b_{11}+b_{22}b_{21} & b_{21}b_{12}+b_{22}^2
  \end{pmatrix} =
\begin{pmatrix}
    0 & 0\\
    0 & 0
  \end{pmatrix}
\) 
Therefore, 
\begin{align}
 &b_{11}^2 = -b_{12}b_{21} \\
 &b_{11}b_{12} = -b_{22}b_{12}\\
 &b_{11}b_{21} = -b_{22}b_{21}\\
 &b_{22}^2 = -b_{21}b_{12}
\end{align}
From the above equations, it can be easily observed that if $b_{11} = 0$, B becomes zero matrix. So assume, $b_{11} \neq 0$. From equations (5) and (8), we can conclude that, $b_{11}^2 = b_{22}^2$ and $b_{12}b_{21} < 0 \implies (b_{12} - b_{21}) \neq 0$. From equations (6) and (7) we get, $(b_{12} - b_{21})(b_{11}+b_{22}) = 0$. Therefore, $b_{11} = -b{22}$. Let $b_{11} = 1, b_{12} = 1$ then $b_{22} = -1$ and $b_{21} = -1$.\\
Therefore, \(
B =
  \begin{pmatrix}
    1 & -1\\
    1 & -1
  \end{pmatrix}
\)  
\end{solution}
\fi

\item[(c)] 
\ifnum\me<2
\begin{solution}
Let \(
C =
  \begin{pmatrix}
    c_{11} & c_{12}\\
    c_{21} & c_{22}
  \end{pmatrix}, 
D =
  \begin{pmatrix}
    d_{11} & d_{12}\\
    d_{21} & d_{22}
  \end{pmatrix}
\) Given that $CD = -DC \implies$
\begin{align}
& 2c_{11}d_{11} = -c_{12}d_{21} - d_{12}c_{21}\\
& c_{11}d_{12} + c_{12}d_{22} = -d_{11}c_{12} - d_{12}c_{22}\\
& c_{21}d_{11} + c_{22}d_{21} = -d_{21}c_{11} - d_{22}c_{21}\\
& 2c_{22}d_{22} = -d_{21}c_{12} - c_{21}d_{12}
\end{align}
Equations (9), (10) $ \implies c_{11}d_{11} = c_{22}d_{22}$\\ Equations (10), (11) $ \implies c_{12}(d_{11}+d_{22}) = -d_{12}(c_{11}+c_{22})$  and $d_{21}(c_{11}+c_{22}) = -c_{21}(d_{11}+d_{22})$

Let $c_{11} = 1, c_{22} = -1, d_{11} = 1$ then $d_{22} = -1$. Now pick $c_{12} = 1$ and $c_{21} = 1$, from equation (9) we get, $-d_{21}-d_{12} = -2$. Pick $d_{21} = -3$ then $d_{12} = 1$.\\
Therefore 
\(
C =
  \begin{pmatrix}
    1 & 1\\
    1 & -1
  \end{pmatrix}, 
D =
  \begin{pmatrix}
    1 & 1\\
    -3 & -1
  \end{pmatrix}
\) where \(
CD =
  \begin{pmatrix}
    -2 & 0\\
    4 & 2
  \end{pmatrix}
\) and \(DC =
  \begin{pmatrix}
    2 & 0\\
    -4 & -2
  \end{pmatrix}
\)


\end{solution}
\fi
\end{itemize}

\newproblem{2}

\ifnum\me<2
\begin{solution}
\begin{itemize}
\item If $A$ is a singular matrix, then for any non-zero vector $x$  we know that $Ax \neq 0$
\item Since $B$ is non singular, there exists a non-zero vector $y$ such that $By = x$. 
\item Therefore, $Ax \neq 0 \implies A(By) \neq 0 \implies (AB)y \neq 0$ where $y$ is a non-zero matrix.
\end{itemize}
$\therefore$ If $A$ and $B$ are non-singular matrices then their product $AB$ is also non-singular.
\end{solution}

\newproblem{3}

\ifnum\me<2
\begin{solution}

Given \(
A =
  \begin{pmatrix}
    1 & 8 & 7\\
    2 & 10 & 8\\
    3 & 12 & 9
  \end{pmatrix}
\)
\begin{align*}
	& a_3 = \lambda_1a_1 + \lambda_2a_2\\
	& \lambda_1 \begin{pmatrix}
    1 \\
    2 \\
    3
  \end{pmatrix} + \lambda_2 \begin{pmatrix}
    8 \\
    10 \\
    12
  \end{pmatrix} = \begin{pmatrix}
    7 \\
    8 \\
    9
  \end{pmatrix} \\
  \implies & \lambda_1 + 8 \lambda_2 = 7 \tag{1}\\
  & 2 \lambda_1 + 10 \lambda_2 = 10 \tag{2}\\
  & 3 \lambda_1 + 12 \lambda_2 = 9 \tag{3}
\end{align*}
Solving equations (1) and (2), we get $\lambda_1 = -1$ and $\lambda_2 = 1$. Substitute these values in (3), we get $-3 + 12 = 9$. Therefore the columns of the matrix are linearly dependent.
\end{solution}

\newproblem{4}

\ifnum\me<2
\begin{solution}

\begin{theorem}
Let $A$ be an $n \times n$ real matrix. If the only solution of $Ay = 0$ is $y = 0$ then prove that there is a unique n-vector $x$ satisfying $Ax = b$ for any nonzero $n$-vector $b$ 
\end{theorem}
\begin{proof}

Let us assume that there are two vectors $x_1$ and $x_2$  where $x_1 \neq x_2$ such that $Ax_1 = b$ and $Ax_2 = b \implies A(x_1-x_2) = 0$. But we know that if $Ay = 0 \implies y$ has to be zero. Therefore, $(x_1 - x_2) = 0 \implies x_1 = x_2$. 

This is a contradiction to the given hypothesis. Thus our assumption was wrong i.e. $x_1$ and $x_2$ should be equal. Hence, there is a unique n-vector $x$ satisfying $Ax = b$ for any nonzero $n$-vector $b$ if the only solution of $Ay = 0$ is $y = 0$
\end{proof}

\begin{theorem}
Let $A$ be an $n \times n$ real matrix. If there is a unique n-vector $x$ satisfying $Ax = b$ for any nonzero $n$-vector $b$ then prove that the only solution of $Ay = 0$ is $y = 0$. 
\end{theorem}
\begin{proof}
Let us assume that there exists a non-zero vector $y$ such that $Ay = 0$. We know that $Ax = b$ where $x$ is a unique $n$-vector and $b$ is a non-zero vector. Adding these two equations, we get $A(x+y) = b$ where $x+y \neq  x$ since $y$ is a non-zero vector.

This is a contradiction to the given hypothesis. Thus our assumption was wrong i.e. $y$ can not be a non-zero vector. Hence, the only solution of $Ay = 0$ is $y = 0$ if there is a unique n-vector $x$ satisfying $Ax = b$ for any nonzero $n$-vector $b$.
\end{proof}
\end{solution}

\newproblem{5}

\begin{itemize}
\item[(a)] 
 
\ifnum\me<2
\begin{solution}

If columns of $A$ are linearly independent then $Az = 0$ only if $z = 0$.

If $Ax = Ay \implies A(x-y) = 0$. Given that the columns of $A$ are linearly independent, $\therefore x-y = 0 \implies x = y$
\end{solution}

\item[(b)]
\ifnum\me<2
\begin{solution}

\(
A =
  \begin{pmatrix}
    -1 & 2 & 3\\
    0 & -2 & 1
  \end{pmatrix}
 x =
  \begin{pmatrix}
    1\\
    2\\
    3 
  \end{pmatrix} \implies Ax =
  \begin{pmatrix}
    12\\
    -1
  \end{pmatrix} \)
  
  Let \(
  y =
  \begin{pmatrix}
    5\\
    \lambda_2\\
    4 
  \end{pmatrix} \implies Ay = \begin{pmatrix}
    12\\
    -1
  \end{pmatrix}
\)

Solve for $\lambda_2$, we get $\lambda_2 = 5/2$
\(\therefore y = y =
  \begin{pmatrix}
    5\\
    5/2\\
    4 
  \end{pmatrix}\)
\end{solution}
\end{itemize}

\newproblem{6}

\ifnum\me<2
\begin{solution}

Rank of $A$ is equal to the maximum number of linearly independent rows and columns of $A$ (which are equal). Thus $r \leq m$ and $r \leq n$. But it is given that $r = m$. Therefore $m \leq n$
\end{solution}

\newproblem{7}

\ifnum\me<2
\begin{solution}

Assume that both (1) and (2) are true
\begin{align*}
y^Tb &= y^T (Ax) & \textit{Using (1)}\\
&= (y^TA)x	& \textit{Associative property of Matrix Multiplication}\\
&= ((y^TA)^T)^Tx\\
&= (A^Ty)^Tx\\
&= 0^Tx & \textit{Using (2)}\\
&= 0
\end{align*}
But from (2), we know that $y^Tb \neq 0$. This is a contradiction. Thus our assumption was wrong, (1) and (2) cannot both be true and thus they are contradictory.
\end{solution}

\newproblem{8}

\begin{itemize}
\item[(a)] 

\ifnum\me<2
\begin{solution}

Given that $x_R$ is the linear combination of columns of $A$. Therefore, there is a $n$ vector $y$ such that $Ay = x_R$.
\begin{align*}
x_R^Tx_N &= (Ay)^Tx_N \\
&= (y^TA^T)x_N \\
&= y^T(A^Tx_N) &\textit{Associative property of Matrix Multiplication}\\
&= y^T0\\
&= 0
\end{align*}
Hence proved
\end{solution}

\item[(b)]
\ifnum\me<2
\begin{solution}

Let us assume that $x_R$ and $x_N$ are not unique. Let $\exists \,\,x_{R_1}, x_{R_2} \in {\Bbb R}(A)$ and $x_{N_1}, x_{N_2} \in {\Bbb N}(A^T)$  where $x_{R_1} \neq x_{R_2}$ and $x_{N_1} \neq x_{N1_2}$ such that
\begin{align*}
& x = x_{R_1} + x_{N_1}\\
& x = x_{R_2} + x_{N_2}\\
& \text{Subtracting these two equations we get}\\
& 0 = (x_{R_1} - x_{R_2}) + (x_{N_1} - x_{N_2})\\
& x_{R_1} - x_{R_2} = -(x_{N_1} - x_{N_2})\tag{1}\\
\end{align*}
But we know that Range space and Null space are closed under addition and subtraction. $\therefore (x_{R_1} - x_{R_2}) \in {\Bbb R}(A)$ and $(x_{N_1} - x_{N_2})\in {\Bbb N}(A^T) \implies (x_{R_1} - x_{R_2}) \perp (x_{N_1} - x_{N_2})$ but from (1) we have,  $(x_{R_1} - x_{R_2}) \parallel (x_{N_1} - x_{N_2}) \implies (x_{R_1} - x_{R_2}) = 0$ and $(x_{N_1} - x_{N_2}) = 0 \implies$\\ $ x_{R_1} = x_{R_2}$ and $x_{N_1} = x_{N_2}$

This is a contradiction. Hence our assumption was wrong. Therefore $x_R$ and $x_N$ are unique.
\end{solution}


\item[(c)]
\ifnum\me<2
\begin{solution}

It is given that $x_R \neq 0$ and $x_N \neq 0$

Assume that $\lambda_1 \neq 0$ and $\lambda_2 \neq 0$ such that 
$\lambda_1 x_R + \lambda_2 x_N = 0 \implies x_R$ and $x_N$ are parallel but we also know that $x_R$ and $x_N$ are perpendicular to each other $\because x_R \in 
{\Bbb R}(A)$ and $x_n \in {\Bbb N}(A^T) \implies$ \\ $ x_R = x_N = 0$. But this a contradiction to the given hypothesis. Hence our assumption is wrong.
Therefore if $x_R$ and $x_N$ are both nonzero then they are linearly independent.  
\end{solution}
\end{itemize}

\newproblem{9}

\begin{itemize}
\item[(a)] 

\ifnum\me<2
\begin{solution}

Given that $C$ is an $m \times n$ matrix with full column rank. Hence all the columns of $C$ are linearly independent. Therefore $Cy = 0 \implies y = 0$. Assume there are two vectors $x_1$ and $x_2$ such that $Cx_1 = d$ and $Cx_2 = d$. $\implies C(x_1 - x_2) = 0$. But we know that if $cy = 0$, $y = 0$. $\therefore x_1 - x_2 = 0 \implies x_1 = x_2$. Therefore $x$ is unique
\end{solution}

\item[(b)]
\ifnum\me<2
\begin{solution}

We know $\exists$ unique $x_R, x_N$ such that $b = x_R + x_N$ and $\exists\,\,y$ such that $x_R = Ay \because x_R \in {\Bbb R}(A)$

$ \implies b = Ax = Ay + x_N \implies Ax-Ay = X_N \implies Ax-Ay \parallel x_N$. 

$Ax-Ay \in {\Bbb R}(A)$ using closure property of Range space $\implies Ax-Ay \perp X_N$

$\therefore Ax-Ay = 0 \implies x=y$. We are sure that $\exists\,\,y$ for the unique $x_R \implies x$ exists too.

$\therefore$ The system $Ax = b$ is compatible $\forall b \in {\Bbb R}^m$
\end{solution}

\item[(c)]
\ifnum\me<2
\begin{solution}

The rank of the $2 \times 4$ matrix cannot be two because if it is 2 then from part (b) the system $Ax = b$ is compatible for every $b \in {\Bbb R}^m$. So choose a matrix whose rank is 1 and columns are linearly independent. It easy to choose such a matrix. Choose the columns such that they are multiples of column 1, so they are linearly dependent.

Let \(
A =
  \begin{pmatrix}
    a_1\,a_2\,a_3\,a_4
  \end{pmatrix} \) where \(a_1 = \begin{pmatrix}
    1\\
    2
  \end{pmatrix} \) $a_2 = 2a_1, a3 = 3a_1, a_4 = 4a_1$.
  
\( \therefore A = \begin{pmatrix}
    1 & 2 & 3 & 4\\
    2 & 4 & 6 & 8
  \end{pmatrix} \implies \)
\( b = Ax = \begin{pmatrix}
    x_1 + 2x_2 + 3x_3 + 4x_4\\
    2x_1 + 4x_2 + 6x_3 + 8x_4
  \end{pmatrix} \in {\Bbb R}(A) \)
  
$\therefore {\Bbb R}(A)$ consists of only those vectors such that $b_2 = 2b_1$. Choose any vector $b$ that violates this property. Let \( b = \begin{pmatrix}
    2\\
    5
  \end{pmatrix} \). Therefore for the chosen $A, b$ the system $Ax = b$ is not compatible.
\end{solution}
\end{itemize}


\newproblem{10}

\begin{itemize}
\item[(a)] 

\ifnum\me<2
\begin{solution}

Let \(
A =
  \begin{pmatrix}
    a_1\,a_2\,a_3\,a_4\,a_5
  \end{pmatrix} \) where $a_i$ is a 2 dimensional vector.
Given that the $rank(A) = 2$ and the 2 $\times$ 2 submatrix consisting of the first two columns of $A$ has rank 1. Therefore $a_1$ and $a_2$ should be linearly dependent on each other. Hence choose $a_2 = 2a_1$ where \(a_1 =
  \begin{pmatrix}
    1\\
    2
  \end{pmatrix} \) and \(a_2 =
  \begin{pmatrix}
    2\\
    4
  \end{pmatrix} \). Clearly $a_1$ and $a_2$ are linearly dependent. Thus $(a_1 a_2)$ has rank 1. Now we need to choose $a_3, a_4, a_5$ such that the maximum number of linear independent columns are only 2. Let \(a_5 =
  \begin{pmatrix}
    5\\
    9
  \end{pmatrix} \). Therefore $a_1, a_5$ and $a_2, a_5$ are linearly independent but $a_1, a_2, a_5$ are linearly dependent $\because a_2 = 2a_1 + 0a_5$. Thus the rank of the matrix constructed thus far is 2. Now choose $a_3, a_4$ such that the maximum number of linearly independent columns still remains to be 2. Simply choose $a_3, a_4$ to be multiples of $a_1$ so that $a_i, a_5$ for $i = 1,\ldots,4$ are linearly independent. Therefore \(A =
  \begin{pmatrix}
    1 & 2 & 3 & 4 & 5\\
    2 & 4 & 6 & 8 & 9
  \end{pmatrix} \)

$a_1$, $a_2$ are linearly dependent \textbf{(confirms (i))}

$a_1 + a_ 4 = a_2 + a_3 + 0a_5 \implies$ Rank cannot be 5

Randomly choose any 4 columns from $A$, if $a_5$ is one of the chosen columns, then simply choose the coefficient of $a_5$ to be 0 and the remaining columns will be linearly dependent on each other as they are all multiples of $a_1$. If $a_5$ is not chosen then it is straight forward that all the columns will be linearly dependent. A similar argument can be given for choosing any 3 columns from $A \implies$ Rank cannot be 3, 4

Choose $a_1, a_5$, $\lambda_1a_1 + \lambda_2a_5 = 0 \implies \lambda_1 = -5\lambda$ and $2\lambda_1 = -9\lambda_2 \implies \lambda_1 = \lambda2 = 0 \therefore$ the rank of the matrix is 2  \textbf{(confirms (ii))}
\end{solution}

\item[(b)]
\ifnum\me<2
\begin{solution}

Let \(x =
  \begin{pmatrix}
    x_1\\
    x_2\\
    x_3\\
    x_4\\
    x_5
  \end{pmatrix} \implies b = Ax = \begin{pmatrix}
    x_1 + 2x_2 + 3x_3 + 4x_4 + 5x_5\\
    2(x_1 + 2x_2 + 3x_3 + 4x_4) + 9x_5
  \end{pmatrix}\)
  
For \(x =
  \begin{pmatrix}
    1\\
    1\\
    1\\
    1\\
    1
  \end{pmatrix} \implies b = \begin{pmatrix}
    15\\
    29
  \end{pmatrix}\)
\end{solution}

\item[(c)]
\ifnum\me<2
\begin{solution}

Let $x_1 + 2x_2 + 3x_3 + 4x_4 = k$, then \(b = Ax = \begin{pmatrix}
    k + 5x_5\\
    2k + 9x_5
  \end{pmatrix} \in range(A)\)
 
 For any 2-vector $b$ the system of linear equations can be solved to get the values of $k$ and $x_5$. Using $k$ choose $x_1,x_2,x_3,x_4$ such that $x_1 + 2x_2 + 3x_3 + 4x_4 = k$ i.e. $\forall b \,\,\exists x$ such that $b = Ax$
 
$\therefore$ It is not possible to find any such $b \not \in range(A)$  
\end{solution}

\item[(d)]
\ifnum\me<2
\begin{solution}

 \(A^T = \begin{pmatrix}
   1 & 2\\
   2 & 4\\
   3 & 6\\
   4 & 8\\
   5 & 9\\
  \end{pmatrix} \implies b = A^Tx = \begin{pmatrix}
   x_1+2x_2\\
   2x_1+4x_2\\
   3x_1+6x_2\\
   4x_1+8x_2\\
   5x_1+9x_2\\
  \end{pmatrix} \in range(A^T)\)

Notice that rows 2 to 4 of $b$ are multiples of row 1. Choose $c$ such that $c$ violates this property.
\( \therefore c = \begin{pmatrix}
   3\\
   7\\
   10\\
   11\\
   14\\
  \end{pmatrix} \not\in range(A^T)\)

\end{solution}
\end{itemize}

\end{document}


