%me=0 student solutions (ps file), me=1 - my solutions (sol file), me=2 - assignment (hw file)
\def\me{0}
\def\num{3}  %homework number
\def\due{Wednesday, October 21}  %due date
\def\course{CSCI-GA.1180-001 
} %course name, changed only once
\def\name{GOWTHAM GOLI (N17656180)}   %student changes (instructor keeps!)
%
\iffalse
INSTRUCTIONS: replace # by the homework number.
(if this is not ps#.tex, use the right file name)

  Clip out the ********* INSERT HERE ********* bits below and insert
appropriate TeX code.  Once you are done with your file, run

  ``latex ps#.tex''

from a UNIX prompt.  If your LaTeX code is clean, the latex will exit
back to a prompt.  To see intermediate results, type

  ``xdvi ps#.dvi'' (from UNIX prompt)
  ``yap ps#.dvi'' (if using MikTex in Windows)

after compilation. Once you are done, run

  ``dvips ps#.dvi''

which should print your file to the nearest printer.  There will be
residual files called ps#.log, ps#.aux, and ps#.dvi.  All these can be
deleted, but do not delete ps1.tex. To generate postscript file ps#.ps,
run

  ``dvips -o ps#.ps ps#.dvi''

I assume you know how to print .ps files (``lpr -Pprinter ps#.ps'')
\fi
%
\documentclass[11pt]{article}
\usepackage{amsfonts}
\usepackage{latexsym}
\usepackage[lined,boxed,linesnumbered]{algorithm2e}
\usepackage{amsmath}
\usepackage{amsthm}
\usepackage{array}
\usepackage{amssymb}
\usepackage{amsthm}
\usepackage{epsfig}
\usepackage{psfrag}
\usepackage{color}
\usepackage{tikz}
\usetikzlibrary{trees}
\usepackage{mathtools}
\usepackage{float}
\setlength{\oddsidemargin}{.0in}
\setlength{\evensidemargin}{.0in}
\setlength{\textwidth}{6.5in}
\setlength{\topmargin}{-0.4in}
\setlength{\textheight}{8.5in}


\usepackage{listings}
\usepackage{color} %red, green, blue, yellow, cyan, magenta, black, white
\definecolor{mygreen}{RGB}{28,172,0} % color values Red, Green, Blue
\definecolor{mylilas}{RGB}{170,55,241}

\newtheorem{theorem}{Theorem}
\newcommand{\handout}[5]{
   \renewcommand{\thepage}{#1, Page \arabic{page}}
   \noindent
   \begin{center}
   \framebox{
      \vbox{
    \hbox to 5.78in { {\bf \course} \hfill #2 }
       \vspace{4mm}
       \hbox to 5.78in { {\Large \hfill #5  \hfill} }
       \vspace{2mm}
       \hbox to 5.78in { {\it #3 \hfill #4} }
      }
   }
   \end{center}
   \vspace*{4mm}
}

\newcounter{pppp}
\newcommand{\prob}{\arabic{pppp}}  %problem number
\newcommand{\increase}{\addtocounter{pppp}{1}}  %problem number

%first argument desription, second number of points
\newcommand{\newproblem}[2]{
\ifnum\me=0
\ifnum\prob>0 \newpage \fi
\increase
\setcounter{page}{1}
\handout{\name, Homework \num, Problem \arabic{pppp}}{\today}{Name: \name}{Due:
\due}{Solutions to Problem \prob\ of Homework \num\ }
\else
\increase
\section*{Problem \num-\prob~(#1) \hfill {#2}}
\fi
}

%\newcommand{\newproblem}[2]{\increase
%\section*{Problem \num-\prob~(#1) \hfill {#2}}
%}

\def\squarebox#1{\hbox to #1{\hfill\vbox to #1{\vfill}}}
\def\qed{\hspace*{\fill}
        \vbox{\hrule\hbox{\vrule\squarebox{.667em}\vrule}\hrule}}
\newenvironment{solution}{\begin{trivlist}\item[]{\bf Solution:}}
                      {\qed \end{trivlist}}
\newenvironment{solsketch}{\begin{trivlist}\item[]{\bf Solution Sketch:}}
                      {\qed \end{trivlist}}
\newenvironment{code}{\begin{tabbing}
12345\=12345\=12345\=12345\=12345\=12345\=12345\=12345\= \kill }
{\end{tabbing}}

%\newcommand{\eqref}[1]{Equation~(\ref{eq:#1})}

\newcommand{\hint}[1]{({\bf Hint}: {#1})}
%Put more macros here, as needed.
\newcommand{\room}{\medskip\ni}
\newcommand{\brak}[1]{\langle #1 \rangle}
\newcommand{\bit}[1]{\{0,1\}^{#1}}
\newcommand{\zo}{\{0,1\}}
\newcommand{\C}{{\cal C}}

\newcommand{\nin}{\not\in}
\newcommand{\set}[1]{\{#1\}}
\renewcommand{\ni}{\noindent}
\renewcommand{\gets}{\leftarrow}
\renewcommand{\to}{\rightarrow}
\newcommand{\assign}{:=}

\newcommand{\AND}{\wedge}
\newcommand{\OR}{\vee}

\newcommand{\Forr}{\mbox{\bf For }}
\newcommand{\To}{\mbox{\bf to }}
\newcommand{\Do}{\mbox{\bf Do }}
\newcommand{\Ifi}{\mbox{\bf If }}
\newcommand{\Thenn}{\mbox{\bf Then }}
\newcommand{\Elsee}{\mbox{\bf Else }}
\newcommand{\Whilee}{\mbox{\bf While }}
\newcommand{\Repeatt}{\mbox{\bf Repeat }}
\newcommand{\Until}{\mbox{\bf Until }}
\newcommand{\Returnn}{\mbox{\bf Return }}
\newcommand{\Swap}{\mbox{\bf Swap }}

\begin{document}

\lstset{language=Matlab,%
    %basicstyle=\color{red},
    breaklines=true,%
    morekeywords={matlab2tikz},
    keywordstyle=\color{blue},%
    morekeywords=[2]{1}, keywordstyle=[2]{\color{black}},
    identifierstyle=\color{black},%
    stringstyle=\color{mylilas},
    commentstyle=\color{mygreen},%
    showstringspaces=false,%without this there will be a symbol in the places where there is a space
    numbers=left,%
    numberstyle={\tiny \color{black}},% size of the numbers
    numbersep=9pt, % this defines how far the numbers are from the text
    emph=[1]{for,end,break},emphstyle=[1]\color{red}, %some words to emphasise
    %emph=[2]{word1,word2}, emphstyle=[2]{style},    
}

\ifnum\me=0
%\handout{PS\num}{\today}{Name: **** INSERT YOU NAME HERE ****}{Due:
%\due}{Solutions to Problem Set \num}
%
%I collaborated with *********** INSERT COLLABORATORS HERE (INDICATING
%SPECIFIC PROBLEMS) *************.
\fi
\ifnum\me=1
\handout{PS\num}{\today}{Name: Yevgeniy Dodis}{Due: \due}{Solution
{\em Sketches} to Problem Set \num}
\fi
\ifnum\me=2
\handout{PS\num}{\today}{Lecturer: Yevgeniy Dodis}{Due: \due}{Problem
Set \num}
\fi

\newproblem{1}

\begin{itemize}
 \item[(a)] 
 
\ifnum\me<2
\begin{solution}
\begin{align*}
||A||_1 &= \max\limits_{1 \leq j \leq n} \sum\limits_{i=1}^{n} |a_{ij}| \implies \text{   (maximum of the absolute column sums)}\\
&= \max \{|1|+|1|, |-2|+|-1|\}\\
&= 3\\
||A||_\infty &=  \max\limits_{1 \leq i \leq n} \sum\limits_{j=1}^{n} |a_{ij}|\implies \text{   (maximum of the absolute row sums)}\\
&= \max \{|1|+|-2|, |1|+|-1|\}\\
&= 3
\end{align*}
\end{solution}
\fi

\item[(b)] 
\ifnum\me<2
\begin{solution}

Let $u = \begin{pmatrix}
u_1\\
u_2
\end{pmatrix}, ||v||_1 = 1 \implies |u_1| + |u_2| = 1$

$Au = \begin{pmatrix}
1 & -2\\
1 & -1
\end{pmatrix} \begin{pmatrix}
u_1\\
u_2
\end{pmatrix} = \begin{pmatrix}
u_1-2u_2\\
u_1-u_2
\end{pmatrix} , ||Au||_1 = ||A||_1 \implies |u_1-2u_2|+|u_1-u_2| = 3$
\end{solution}
Assume, $u_1 \geq 0$ and $u_2 < 0 \implies u_1 - 2u_2 >0$ and $u_1 - u_2 > 0$

$\therefore u_1 - u_2 = 1$ and $(u_1-2u_2)+(u_1-u_2) = 3 \implies 2u_1-3u_2=3$. Solving these two equations we get $u_1 = 0, u_2 = -1 \implies$
$ u = \begin{pmatrix}
0\\
-1
\end{pmatrix}$
\fi

\item[(c)] 
\ifnum\me<2
\begin{solution}
Let $v = \begin{pmatrix}
v_1\\
v_2
\end{pmatrix} , ||v||_\infty = 1 \implies \max \{|v_1|, |v_2|\} = 1$

$Av = \begin{pmatrix}
1 & -2\\
1 & -1
\end{pmatrix} \begin{pmatrix}
v_1\\
v_2
\end{pmatrix} = \begin{pmatrix}
v_1-2v_2\\
v_1-v_2
\end{pmatrix} , ||Av||_\infty = ||A||_\infty \implies \max \{|v_1-2v_2|, |v_1-v_2|\} = 3$

$\max \{|v_1|, |v_2|\} = 1 \implies $ one of $v_1, v_2$ has to be either 1 or -1. Let $v_1 = 1$

$ \therefore \max \{|1-2v_2|, |1-v_2|\} = 3$. We easily see that $v_2 = -1$ satisfies this equation

$\therefore v = \begin{pmatrix}
1\\
-1
\end{pmatrix} $
\end{solution}
\fi
\end{itemize}

\newproblem{2}

\ifnum\me<2
\begin{solution}

Since $A$ is a non-singular matrix, $A^{-1}$ will exist
\begin{align*}
Ax &= b\\
\implies x &= A^{-1}b\\
\implies ||x|| &\leq ||A^{-1}||\,\,||b||\\
\implies ||A^{-1}|| &\geq \frac{||x||}{||b||}\\
\implies ||A^{-1}|| &\geq \frac{1}{10^{-6}}\\
\implies ||A^{-1}|| &\geq 10^6
\end{align*}
$\therefore cond(A) = ||A|| \,\, ||A^{-1}|| \geq 10^6 \implies$
$cond(A)$ is large therefore $A$ is ill-conditioned

\end{solution}


\newproblem{3}

\begin{itemize}
\item[(a)] 
\ifnum\me<2
\begin{solution}

Using GEPP, $M_1 = \begin{pmatrix}
1 & 0\\
-m_{21} & 1
\end{pmatrix} = \begin{pmatrix}
1 & 0\\
-a_{21}/a_{11} & 1
\end{pmatrix} = \begin{pmatrix}
1 & 0\\
-0.8736 & 1
\end{pmatrix}$

\begin{align*}
M_1Ax &= M_1b\\
\implies \begin{pmatrix}
1 & 0\\
-0.8736 & 1
\end{pmatrix} \begin{pmatrix}
0.625 & 0.4376\\
0.546 & 0.3823
\end{pmatrix}\begin{pmatrix}
x_1\\
x_2
\end{pmatrix} &= \begin{pmatrix}
1 & 0\\
-0.8736 & 1
\end{pmatrix} \begin{pmatrix}
1.0626\\
0.9283
\end{pmatrix}\\
\implies \begin{pmatrix}
0.625 & 0.4376\\
0 & 0.00001264
\end{pmatrix} \begin{pmatrix}
x_1\\
x_2
\end{pmatrix} &=  \begin{pmatrix}
1.0626\\
0.00001264
\end{pmatrix} \\
\implies x_2 &= 1\\
\implies 0.625x_1 + 0.4376 &= 1.0626\\
\implies x_1 &= 1\\
\therefore x = \begin{pmatrix}
1 \\
1
\end{pmatrix}
\end{align*}
\end{solution}

\item[(b)]
\ifnum\me<2
\begin{solution}
\begin{lstlisting}[frame=single]
format long e;
A = [0.625 0.4376;
     0.546 0.3823];
b = [1.0626; 0.9283];
x_star = [1;1];
x_tilda = A\b
\end{lstlisting}
$\widetilde{x} = \begin{pmatrix}
9.999999999975465e-01\\
1.000000000003504e+00
\end{pmatrix}$
\begin{itemize}
\item[i.]

\begin{lstlisting}[frame=single]
d = x_tilda - x_star
norm_d = norm(d)
\end{lstlisting}
\begin{align*}
d &= \begin{pmatrix}
-2.453481862119133e-12\\
3.504085910321919e-12
\end{pmatrix}\\
||d|| &= 4.277638520803758e-12
\end{align*}
Since $||d||$ is almost zero, it means that $\widetilde{x}$ and $x$ are almost equal
\item[ii.]

\begin{lstlisting}[frame=single]
r_star = b - A*x_star
norm_r_star = norm(r_star)
\end{lstlisting}
\begin{align*}
r^* &= \begin{pmatrix}
0\\
0
\end{pmatrix}\\
||r^*|| &= 0
\end{align*}
Since $||r^*||$ is  zero, it means that $b$ and $Ax^*$ are exactly equal
\item[iii.]

\begin{lstlisting}[frame=single]
r_tilda = b - A*x_tilda
norm_r_tilda = norm(r_tilda)
\end{lstlisting}
\begin{align*}
 \widetilde{r} &= \begin{pmatrix}
0\\
0
\end{pmatrix}\\
||\widetilde{r}|| &= 0
\end{align*}
Since $||\widetilde{r}||$ is  zero, it means that $b$ and $A\widetilde{x}$ are exactly equal. (Although they aren't exactly equal the difference is so low that matlab rounded it off to 0)
\end{itemize}
\end{solution}


\item[(c)]
\ifnum\me<2
\begin{solution}
\begin{lstlisting}[frame=single]
x_hat = [-27.678;41.958];
r_hat = b - A*x_hat;
E = (1/((transpose(x_hat))*x_hat))*(r_hat*transpose(x_hat))
b1 = (A+E)*x_hat
\end{lstlisting}

\begin{align*}
E &= \begin{pmatrix}
-5.797322035777278e-06   &  8.788353131625949e-06\\
     6.069003037814685e-07 &   -9.200203391163686e-07
\end{pmatrix}\\
b1 &= \begin{pmatrix}
1.062600000000000e+00\\
     9.282999999999986e-01
\end{pmatrix} \approx b
\end{align*}
\end{solution}


\item[(d)]
\ifnum\me<2
\begin{solution}
\begin{lstlisting}[frame=single]
norm_x_hat_x_star = norm(x_hat - x_star)
norm_x_hat_x_tilda = norm(x_hat - x_tilda)
\end{lstlisting}
\begin{align*}
||\hat{x}-x^*|| &= 4.999985447978824e+01\\
||\hat{x}-\widetilde{x}|| &= 4.999985447978396e+01
\end{align*}
We can see that $||\hat{x}-\widetilde{x}|| \approx ||\hat{x}-x^*||$ and $||\hat{x}-\widetilde{x}|| < ||\hat{x}-x^*||$ this means that
 $\hat{x}$ is more closer to $\widetilde{x}$ than $\hat{x}$ is closer to $x^*$. 
\end{solution}


\item[(e)]
\ifnum\me<2
\begin{solution}

\begin{itemize}
\item $\hat{r} = \begin{pmatrix}
5.292000000030050e-04\\
    -5.539999999670808e-05
\end{pmatrix}$

\item $E = \begin{pmatrix}
-5.797322035777278e-06   &  8.788353131625949e-06\\
6.069003037814685e-07 &   -9.200203391163686e-07
\end{pmatrix}$

\item $||E|| = 1.058578570328090e-05$
\item $||A|| = 1.013108113647895e+00$
\end{itemize}
$||E|| << ||A||$. Therefore we can imply that $||E||$ is small
\end{solution}


\item[(f)]
\ifnum\me<2
\begin{solution}
\begin{lstlisting}[frame=single]
x_bar = (A+E)\b
\end{lstlisting}
\begin{align*}
\bar{x} = \begin{pmatrix}
 -2.767800001642768e+01\\
     4.195800002346206e+01
\end{pmatrix}
\end{align*}
\end{solution}

\item[(g)]
\ifnum\me<2
\begin{solution}
\begin{lstlisting}[frame=single]
norm_x_bar_x_hat = norm(x_bar - x_hat)
\end{lstlisting}
\begin{align*}
||\bar{x}-\hat{x}|| = 2.864152054095992e-08
\end{align*}
Yes the norm is small i.e $\bar{x}$ is close to $\hat{x}$
\end{solution}

\item[(h)]
\ifnum\me<2
\begin{solution}

Yes it can be said that $\hat{x}$ is close to the exact solution of a system that is close to the original system because $||\hat{x} - x^*|| = 4.999985447978824e+01 \approx ||\hat{x} - \widetilde{x}|| = 4.999985447978396e+01 $

\end{solution}
\end{itemize}

\newproblem{4}

\begin{itemize}

\item[(a)]
\ifnum\me<2
\begin{solution}

\begin{align*}
cond(BC) &= ||BC|| \cdot ||(BC)^{-1}||\\
&= ||BC|| \cdot ||C^{-1}B^{-1}||\\
&\leq ||B|| \cdot ||C|| \cdot ||C^{-1}|| \cdot ||B^{-1}||\\
&\leq  ||B|| \cdot ||B^{-1}|| \cdot ||C|| \cdot ||C^{-1}||\\
\implies cond(BC) & \leq cond(B) cond(C)
\end{align*}

\end{solution}

\item[(b)]
\ifnum\me<2
\begin{solution}

According to SVD, we know that if $A = USV^T$ then $||A||_2 = \sigma_1$ and $||A^{-1}||_2 = 1/\sigma_n $
\begin{align*}
A &= USV^T\\
\implies A^T &= VS^TU^T\\
\implies A^T &= VSU^T
\end{align*}
This looks very much like the SVD of $A$ except that the orthogonal matrices are interchanged but $S$ remains the same while holding the crucial property that $ \sigma_1 \geq \dots \geq \sigma_n$. Therefore the the maximum in $||A^T||_2 =  \max\limits_{||x||_2 \neq 0} \frac{||A^Tx||_2}{||x||_2} = \max\limits_{||y||_2 = 1} ||A^Ty||_2$ is achieved when $x$ is a multiple of $u_1$. As a result, the unit-norm of vector $y$ for which $||A^T||_2$ is maximized is $u_1$

$ \therefore \max\limits_{||y||_2 = 1} ||A^Ty||_2 = ||A^Tu_1||_2 = \sigma_1 = ||A^T||_2$.  Similar to the arguments given to $||A^{-1}||_2$ in the lecture notes, we can prove that $||(A^{T})^{-1}||_2 = 1/\sigma_n$ as $A$ and $A^T$ are structurally similar in SVD. 
\begin{align*}
cond(A^T) &= ||A^T||_2 \cdot ||(A^T)^{-1}||_2\\
&= \sigma_1/\sigma_n\\
&= cond(A)
\end{align*}
\end{solution}

\item[(c)]
\ifnum\me<2
\begin{solution}
\begin{align*}
cond(A) &= ||A||_1 \cdot ||A^{-1}||_1 \tag{1}\\
cond(A^T) &= ||A^T||_1 \cdot ||(A^T)^{-1}||_1\\
\implies cond(A^T) &= ||A^T||_1 \cdot ||(A^{-1})^{T}||_1 \tag{2}
\end{align*}
If we compare (1) and (2), the first term in both the expressions are $||A||_1$ and $||A^T||_1$. We know that the one norm is the maximum column sum of $A$. Since the columns of $A^T$ are the rows of $A$, these two terms will be equal only if the maximum column sum of $A$ = maximum row sum of $A$ i.e.  $||A||_1 = ||A^T||_1 \Leftrightarrow ||A||_1 = ||A||_{\infty}$.

The similar argument holds for the second term in (1) and (2) i.e.\\ $||A^{-1}||_1 = |(A^{-1})^{T}||_1 \Leftrightarrow ||A^{-1}||_1 = ||A^{-1}||_{\infty}$

$\therefore cond(A) = cond(A^T)$ only when both the above conditions hold or if the conditions fail but the products become equal. It easy to choose any $A$ such that these conditions are violated.

Let $A = \begin{pmatrix}
1 & 3 & 4\\
0 & 5 & -8\\
6 & 3 & 0
\end{pmatrix} \implies ||A||_1 = 12, ||A||_{\infty} = 13 \implies ||A||_1 \neq ||A||_{\infty}$.
\begin{lstlisting}[frame = single]
A = [1 3 4; 0 5 -8; 6 3 0];
x = cond(A,1)
y = cond(transpose(A),1)
\end{lstlisting}
\begin{align*}
cond(A) = x &=  5.100000000000001e+00\\
cond(A^T) = y &= 4.333333333333334e+00\\
\implies cond(A) \neq cond(A^T)
\end{align*}
\end{solution}

\end{itemize}
\newproblem{5}

\begin{itemize}
\item[(a)] 
 
\ifnum\me<2
\begin{solution}

 $ \because A$ is 3 $\times 2$ matrix, $r(A) \leq 2$. But clearly $a_2 = 2a_1$. $\therefore r(A) \neq 2 \implies r(A) = 1$
\end{solution}

\item[(b)]
\ifnum\me<2
\begin{solution}

${\rm I\!R(A)}$ includes all the vectors of the form
\begin{align*}
A\begin{pmatrix}
\lambda_1\\
\lambda_2 \end{pmatrix} = \begin{pmatrix}
2 & 4\\
1 & 2\\
1& 2
\end{pmatrix} \begin{pmatrix}
\lambda_1\\
\lambda_2 \end{pmatrix} = \begin{pmatrix}
2\lambda_1+4\lambda_2\\
\lambda_1 + 2\lambda_2\\
\lambda_1 + 2\lambda_2
\end{pmatrix} \in {\rm I\!R(A)}
\end{align*}
Let $\alpha = \lambda_1 + 2\lambda_2$ where $\alpha$ is a scalar then
$\lambda = \begin{pmatrix}
2\alpha\\
\alpha\\
\alpha
\end{pmatrix} \in {\rm I\!R(A)}$
\end{solution}

\item[(c)]
\ifnum\me<2
\begin{solution}

${\rm I\!N(A^T)}$ has the dimension $m-r(A) = 3-1 = 2$. ${\rm I\!R(A^T)}$ contains all the vectors of the form $z$ such that
\begin{align*}
A^Tz &= 0\\
\implies \begin{pmatrix}
2 & 1 & 1\\
4 & 2 & 2
\end{pmatrix} \begin{pmatrix}
z_1\\
z_2\\
z_3
\end{pmatrix} &= 0
\implies 2z_1 + z_2 + z_3 &= 0\\
\implies z &= \begin{pmatrix}
\gamma_1\\
\gamma_2\\
-2\gamma_1 - \gamma_2
\end{pmatrix} \in {\rm I\!N(A^T)} \text{   where } \gamma_1, \gamma_2 \text{ are any scalars}
\end{align*}
Let $\gamma_1 = 1, \gamma_2 = 1 \implies$ $z_1 = \begin{pmatrix}
1\\
1\\
-3
\end{pmatrix}$

Let $\gamma_1 = 0, \gamma_2 = 1 \implies$ $z_2 = \begin{pmatrix}
0\\
1\\
-1
\end{pmatrix}$
\begin{align*}
\lambda_1z_1 + \lambda_2z_2 &= 0\\
\implies \lambda_1 = 0, \lambda_1 + \lambda_2 &= 0, 3\lambda_1 + \lambda_2 = 0\\
\implies \lambda_1 &= \lambda_2 = 0
\end{align*}
$\therefore z_1, z_2$ are linearly independent 
\end{solution}

\item[(d)]
\ifnum\me<2
\begin{solution}
Let $b_R = \begin{pmatrix}
2\alpha\\
\alpha\\
\alpha
\end{pmatrix} $, $b_N =\begin{pmatrix}
\gamma_1\\
\gamma_2\\
-2\gamma_1 - \gamma_2
\end{pmatrix}$ 
\begin{align*}
\begin{pmatrix}
3\\
2\\
1
\end{pmatrix} & =\begin{pmatrix}
2\alpha\\
\alpha\\
\alpha
\end{pmatrix} + \begin{pmatrix}
\gamma_1\\
\gamma_2\\
-2\gamma_1 - \gamma_2
\end{pmatrix}\\
\implies 2\alpha + \gamma_1 &= 3\\
\alpha + \gamma_2 &= 2\\
\alpha -2\gamma_1 -\gamma_2 &= 1
\end{align*}
Solving these three equations, we get 
$b_R = \begin{pmatrix}
3\\
3/2\\
3/2
\end{pmatrix}$ and $b_N = \begin{pmatrix}
0\\
1/2\\
-1/2
\end{pmatrix}$
\end{solution}

\item[(e)]
\ifnum\me<2
\begin{solution}
${\rm I\!N(A)}$ has dimension $n-r(A) = 1$. ${\rm I\!N(A)}$ contains all the vectors of the form $q$ such that
\begin{align*}
Aq &= 0\\
\implies \begin{pmatrix}
2 & 4\\
1 & 2\\
1 & 2
\end{pmatrix} \begin{pmatrix}
q_1\\
q_2
\end{pmatrix} &= 0
\implies q_1 + 2q_2 = 0\\
\implies q = \begin{pmatrix}
-2q_2\\
q_2
\end{pmatrix} \in {\rm I\!N(A)}
\end{align*}
\end{solution}

\item[(f)]
\ifnum\me<2
\begin{solution}
\begin{align*}
Av &= b_R\\
\implies \begin{pmatrix}
2 & 4\\
1 & 2\\
1& 2
\end{pmatrix} \begin{pmatrix}
v_1\\
v_2 \end{pmatrix} &= \begin{pmatrix}
3\\
3/2\\
3/2
\end{pmatrix} \implies v_1 + 2v_2 = 3/2\\
\implies v &= \begin{pmatrix}
3/2-2v_2\\
v_2
\end{pmatrix}
\end{align*}
Let $v_2 = 1/2 \implies v_1 = 1/2 \implies v = \begin{pmatrix}
1/2\\
1/2
\end{pmatrix}$
\end{solution}

\item[(g)]
\ifnum\me<2
\begin{solution}
\begin{itemize}
\item Let $v_2 = 1/2 \implies v_1 = 1/2 \implies v = \begin{pmatrix}
1/2\\
1/2
\end{pmatrix}$

Let $q_2 = 1 \implies q = \begin{pmatrix}
-2\\
1
\end{pmatrix}$

$ \implies b_A = \begin{pmatrix}
-3/2\\
3/2
\end{pmatrix}$

\item Let $v_2 = 1 \implies v_1 = -1/2 \implies v = \begin{pmatrix}
-1/2\\
1
\end{pmatrix}$

Let $q_2 = -1 \implies q = \begin{pmatrix}
2\\
-1
\end{pmatrix}$

$ \implies b_A = \begin{pmatrix}
3/2\\
0
\end{pmatrix}$
\end{itemize}
\end{solution}

\end{itemize}


\end{document}


